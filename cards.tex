% Code modified from https://tex.stackexchange.com/questions/243740/print-double-sided-playing-cards

\documentclass[a4paper]{article}
\usepackage{graphicx}
\usepackage[export]{adjustbox}
%\usepackage{color}
\usepackage{amsmath, amssymb}
\usepackage{tikz}
\usetikzlibrary{patterns, shadows, positioning}
\usepackage{pifont}
\usepackage{fourier-orns}
\usepackage[hmargin=0mm,vmargin=0mm, letterpaper]{geometry}
\usepackage{tabularx}
\usepackage{epsdice}

%\setlength{\parindent}{0cm}

\definecolor{titlebg}{rgb}{30, 30 , 30}

\newcommand{\icon}{$\alpha$}
\newcommand{\cardtype}{cardtype}
\newcommand{\cardtitle}{cardtitle}
\newcommand{\cost}{cost}
\newcommand{\flavortext}{flavortext}
\newcommand{\cardcontent}{cardcontent}
\newcommand{\cardid}{cardid}
\newcommand{\cardimg}{rainier.jpg}

%   TikZ/PGF Settings für die Karten
\def\cardwidth{2.5in}
\def\cardheight{3.5in}
\def\imagewidth{\cardwidth}
\def\imageheight{\cardheight}
\pgfmathsetmacro{\stripwidth}{0.7pt}
\pgfmathsetmacro{\strippadding}{0.2pt}
\pgfmathsetmacro{\textpadding}{0.1pt}
\pgfmathsetmacro{\titley}{\cardheight-\strippadding-1.5*\textpadding-0.5*\stripwidth}

\def\shapeCard{(0,0) rectangle (\cardwidth, \cardheight)}
\def\shapeLeftStripLong{(0,0) rectangle 
    (\stripwidth,\cardheight)}
\def\shapeLeftStripShort{(\strippadding,\cardheight-\strippadding-1) rectangle 
    (\strippadding+\stripwidth,\cardheight+0.2)}
\def\shapeRightStripShort{(\cardwidth-\stripwidth-\strippadding,
    \cardheight-\strippadding-1) rectangle (\cardwidth-\strippadding,   
    \cardheight+0.2)}
\def\shapeRightStripLong{(\strippadding,-0.2) rectangle 
    (\strippadding-\stripwidth,\cardheight-\strippadding-\strippadding-1)}
\def\shapeTitleArea{(2*\strippadding+\stripwidth,\cardheight-\strippadding) 
    rectangle (\cardwidth-2*\strippadding-\stripwidth,\cardheight-2*\stripwidth)}
\def\shapeContentArea{(2*\strippadding+\stripwidth,0.5*\cardheight) rectangle 
    (\cardwidth+0.2,-0.2)}

\tikzset{%
  cardimage/.style={path picture={
      \node[below=-1.5mm] at (0.5*\cardwidth,\cardheight) {
          \includegraphics[width=\imagewidth,keepaspectratio]{#1}
      };}},
  strips/.style={olive!100},
  strip font/.style={font=\LARGE, text=white}
}

\newcommand{\cardbackground}[1]{
    \draw[cardimage=#1] \shapeCard;}

\newcommand{\frontcard}[1][]{#1
\begin{tikzpicture}
  \cardbackground{\cardimg}        

  % card type strip
  \fill[strips] \shapeLeftStripLong node[rotate = 90, above left, strip font] 
       {\uppercase{\cardtype}};
       
       % card content
       \node[font=\normalsize, rotate = 90,
         fill=white] 
       at (5,5) 
       { \cardcontent };                      
       
\end{tikzpicture}}


\newcommand{\trailcard}[6]{
\begin{tikzpicture}
  %begin card creation

      %\draw [step=.5, help lines] (0,0) grid (\cardwidth,\cardheight);
    % draw card boundries and clip corners
    \draw[black] \shapeCard;

    \def\offset{0.35}
    \begin{scope}
      \node (A) at (#1-\offset,0.75) {\Huge\epsdice{#1}};
      \node (B) at (#2-\offset,2.25) {\Huge\epsdice{#2}};
      \node (C) at (#3-\offset,3.75) {\Huge\epsdice{#3}};
      \node (D) at (#4-\offset,5.25) {\Huge\epsdice{#4}};
      \node (E) at (#5-\offset,6.75) {\Huge\epsdice{#5}};
      \node (F) at (#6-\offset,8.25) {\Huge\epsdice{#6}};

      \draw (A) -- (B) -- (C) -- (D) -- (E) -- (F);
    \end{scope}
     %%text width=(\cardwidth-2*\strippadding-\stripwidth-2*\textpadding-0.3)*1cm] 
     %at (2*\strippadding+\stripwidth+\textpadding, 0.0) 
     %{ \cardid };
     
\end{tikzpicture}}

\begin{document}

% https://tex.stackexchange.com/questions/56101/spaces-between-rows-and-cols-in-a-table-better-ways-than-mine
\setlength\tabcolsep{0pt}
\renewcommand{\arraystretch}{0}

\begin{table}[h]
  \centering
  \begin{tabular}{|c|c|c|}
  \frontcard[\renewcommand{\cardtype}{\LARGE Mt. Rainier}%
    \renewcommand{\cardcontent}{\LARGE Must rest after using stamina}%
    \renewcommand{\cardimg}{rainier.jpg}] &
  \frontcard[
    \renewcommand{\cardtype}{\LARGE Great Sand Dunes}%
    \renewcommand{\cardcontent}{\LARGE Reduce stamina by one}%
    \renewcommand{\cardimg}{dunes.jpg}] &
  \frontcard[%
    \renewcommand{\cardtype}{\LARGE Joshua Tree}%
    \renewcommand{\cardcontent}{\LARGE May not save stamina $=$ 6}%
    \renewcommand{\cardimg}{joshua_tree.jpg }] \\
  \hline
  \frontcard[
    \renewcommand{\cardtype}{\LARGE Pipestone}%
    \renewcommand{\cardcontent}{\LARGE Die may be $>=$ terrain}%
    \renewcommand{\cardimg}{pipestone.jpg}] &
  \frontcard[
    \renewcommand{\cardtype}{\LARGE Black Hills}%
    \renewcommand{\cardcontent}{\LARGE Must rest on triplets}%
    \renewcommand{\cardimg}{black_hills.jpg}] &
  \frontcard[
    \renewcommand{\cardtype}{\LARGE Cedar Breaks}%
    \renewcommand{\cardcontent}{\LARGE May save two stamina}%
    \renewcommand{\cardimg}{cedar.jpg}]  \\
    \hline
  \frontcard[
    \renewcommand{\cardtype}{\LARGE Badlands}%
    \renewcommand{\cardcontent}{\LARGE Terrains 4--6 reduced to 3}%
    \renewcommand{\cardimg}{badlands.jpg}] &
  \frontcard[
    \renewcommand{\cardtype}{\LARGE Grand Canyon}%
    \renewcommand{\cardcontent}{\LARGE Complete all 5 path cards}%
    \renewcommand{\cardimg}{grand_canyon.jpg}] &
  \frontcard[
    \renewcommand{\cardtype}{\LARGE Pikes Peak}%
    \renewcommand{\cardcontent}{\LARGE Ride to the top for 5 rests}%
    \renewcommand{\cardimg}{pikes_peak.jpg}]  \\ 
  \end{tabular}
  \end{table}

\begin{table}[h]
  \centering
  \begin{tabular}{|c|c|c|}
    %\hline \\
    \trailcard{6}{1}{2}{5}{4}{3} &
    \trailcard{5}{3}{1}{2}{6}{4} &
    \trailcard{2}{1}{5}{3}{4}{6} \\
    %\hline
    \trailcard{4}{5}{2}{3}{6}{1} &
    \trailcard{2}{4}{1}{3}{6}{5} &
    \trailcard{2}{3}{6}{1}{4}{5} \\
    %\hline
    \trailcard{1}{4}{2}{5}{3}{6} &
    \trailcard{3}{1}{5}{4}{2}{6} &
    \trailcard{6}{2}{5}{4}{3}{1} %\\
    %\hline
  \end{tabular}%
  \hskip\headheight
\end{table}


\end{document}
